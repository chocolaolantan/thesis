\chapter{実験データ詳細}
\label{app_b}

実験5.1で対象としたデータの各文は以下となる.\\

\textbf{要約対象文:[社説] 4者会談 米朝の外交努力まだ不足}\\\\
\textbf{01:}南北朝鮮と米中4カ国による朝鮮問題4者会談が,21日からジュネーブで再開される.\\
\textbf{02:}会談では,朝鮮半島の恒久的平和体制の確立が,主要議題になる.\\
\textbf{03:}会談の成功と継続を期待したい.\\
\textbf{04:}4者会談は,1996年4月に韓国の金泳三前大統領とクリントン米大統領が共同提案した.\\
\textbf{05:}朝鮮民主主義人民共和国(北朝鮮)は,当初韓国との同等な資格での会談に消極的な姿勢を見せていたが,最終的には応じることになった.\\
\textbf{06:}朝鮮半島は,国際法上はなお戦争状態にある.\\
\textbf{07:}ただ,現在は戦闘を停止した「休戦」が続いているに過ぎない.\\
\textbf{08:}これを平和の状態に変えるには「平和協定」の締結が不可欠だ.\\
\textbf{09:}ところが,当事者の思惑が異なり,そう簡単に事が運ばないのだ.\\
\textbf{10:}最大の争点は,だれが平和協定の当事者であるかの駆け引きである.\\
\textbf{11:}北朝鮮は,休戦協定の当事者は北朝鮮と米国であり,韓国は当事者にはならないと主張している.\\
\textbf{12:}一方,米韓両国は「韓国も平和協定の当事者として署名すべきである」と主張して譲らない.\\
\textbf{13:}なぜ,北朝鮮は韓国を当事者として認めないのか.\\
\textbf{14:}北朝鮮は,なお公式には韓国の存在を認めていないからである.\\
\textbf{15:}朝鮮半島における唯一合法的な国家は北朝鮮であるとの立場を捨てていない.\\
\textbf{16:}北朝鮮はまた,4者会談で在韓米軍撤退も議題にするよう求めている.\\
\textbf{17:}米韓両国は,これを交渉議題にすることに反対している.\\
\textbf{18:}4者会談のもう一つの問題は,韓国の金大中(キムデジュン)政権にとっては前政権が提案した構想である事実だ.\\
\textbf{19:}そのためか,金大中政権にはもうひとつ熱心になれない様子がうかがえる.\\
\textbf{20:}ただ,南北対話が中断しているため4者会談の場でしか,南北の接触を実現できない事情も韓国側にはある.\\
\textbf{21:}米国内では,4者会談を含む米朝交渉について「譲歩し過ぎだ」との批判が絶えない.\\
\textbf{23:}交渉が行き詰まっているにもかかわらず,北朝鮮に食糧を支援するなど,米国は弱腰だというのである.\\
\textbf{23:}この批判は,決して的外れの指摘ではない.\\
\textbf{24:}米国のカートマン担当大使の外交戦術に,やや問題が残るのも否定できない事実だ.\\
\textbf{25:}北朝鮮の外交に対する理解が不足しているうえ,平壌内部の政策決定過程についての情報も十分に入手できていないからだ.\\
\textbf{26:}北朝鮮の外交には,瀬戸際まで行かないと平壌の上層部を納得させ,譲歩を引き出せない性格がつきまとう.\\
\textbf{27:}交渉が決裂するかもしれない状況をうまく演出できないと,平壌から譲歩を引き出すのは難しい.\\
\textbf{28:}それだけに,北朝鮮との交渉には「決裂の覚悟」とその演出が常に求められるのである.\\
\textbf{29:}ところが,米国の対応には「決裂だけは避けたい」との思いがあふれ出ている.\\
\textbf{30:}これでは,北朝鮮との交渉では,一方的に譲歩し続けるしかなくなる.\\
\textbf{31:}一方,北朝鮮の対応にも問題がある.\\
\textbf{32:}米議会の影響力と米世論が怒った時の怖さを,十分に理解していないのだ.\\
\textbf{33:}会談の当事国には,これまでの経験と教訓を十分に生かし,朝鮮半島の休戦状態を一日も早く平和体制に移行させるために,知恵を出し合う努力を求めたい.\\
