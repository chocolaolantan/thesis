\chapter{序論}

\section{はじめに}
SNSなどのサービスが普及し、以前にも増して世界中の人々がインターネット上にテキストデータを投稿するようになった。こうして集められるデータは、その人の趣向、動向、人と為りや、世界の動向を知る手掛かりとなるだろう。しかし、インターネット上にあるテキストデータは膨大なものになっており、ユーザーが本当に欲しい情報は見えにくく、扱いづらいものとなってしまっている。\\
膨大なデータを有効活用していくためには、効率的な分類、加工をしていかなければならないが、データのすべてを人手で解析し、まとめることは非常に困難である。そこで、計算機を利用してこのテキストデータを処理することがさまざまな分野で考えられている。
\\
テキストデータを計算機で処理するにあたって、テキストを何らかの方法で数値化する必要がある。そもそもテキストを構成する文字自体は単なる記号であり、また、計算機が文字データを識別するために、文字一つ一つに事前に割り当てられている数値には、記号識別以上の意味がない。\\
そこで、テキストを計算機で処理するために、いかにして意味を持つ数値で表現し、どのように利用するか、が課題となっている。
\\
\section{研究背景}
テキストを、意味を持つ数値で表現する手法について、現在まででいくつかの方法が研究されている。中でも2013年に発表されたword2vecは、ニューラルネットワークによって学習した単語の意味表現ベクトルで、単語の和、差の計算ができるようになり話題となった。\\
\\
このベクトルの加算減算が可能になったこと、形態素ベクトルの位置関係に相似性が見られることは、主成分分析、t-SNEなどで次元を削減し、可視化したデータから見て取れる。
\\
\section{本研究の目的}
本研究では、word2vecにより得たベクトルデータを可視化することなしに、語の相似関係を抽出することを目的としている。
\\
\section{本論文の構成}
はじめに本章は序論であり、研究背景に関して述べた。\\
本稿2章では、本研究で用いる単語の意味表現についての説明を述べた。
3章では、本研究で用いるベクトルデータを出力するword2vecについて述べた。\\
4章では、word2vecにより得られたデータの解析に用いる、梅山氏のグラフノードマッチング問題に関する提案手法と、ハンガリアン法について述べた。\\
5章で、本研究で行った実験の流れについて説明し、結果を検証した。\\
最後に6章では、本研究における結論と、今後の展望について考慮していることを述べて最後のまとめとした。\\
