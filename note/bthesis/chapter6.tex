\chapter{結び}
本章で、本研究を振り返っての考察と、今後の展望を述べて、研究の結びとする。

\section{実験結果の考察}
前章(\ref{ume_yq})にて梅山氏の手法の有効性とword2vecの学習モデルの妥当性をチェックしたが、この実験を通して、word2vecの出力ベクトルの和差で獲得できる関係性については、今回の実験で利用した、余弦類似度で作成したコスト行列にハンガリアン法を適用する手法により抽出できることが期待される。

また、後にk-meansにより分割したクラスターを重心に関する、クラスター同士のノードマッチングによってささやかながらも関係性が見て取れた。

本研究ではベクトルの和差によって抽出できる単語の類似性については、ハンガリアン法を用いたノードマッチングの手法により抽出できることが期待される。また、近傍単語同士の対応関係についても、クラスタリングを用いることにより単語対の作成時の制限を緩めることで、ある程度の抽出ができる兆候を見ることができた。

しかし、いずれの観点についても検証が不十分であり、ハンガリアン法を用いたノードマッチングの強みのひとつでもある、回転したグラフや拡大・縮小されたグラフ構造においての同型性も抽出できる利点をword2vecの出力ベクトル空間においても有効活用できるかどうかの検証ができていない。

また、単語対から取得するn個の単語同士の対応関係の抽出についても、一方でn個の単語を取得してから、もう一方のn個の単語の取得仕方を決めるなどの工夫をすることができなかった。

今回確認した2点については、あくまでも期待できる程度のものであり、今後の検討を要するものである。

\section{期待する展望}
今回の研究の過程で、グラフの同型性判定を行った。word2vecの出力ベクトルにおいても対応関係の抽出が可能であると期待されたことから、単語ベクトルを用いて文グラフを作成し、文書内に存在する文グラフ同士の同型性をみることで、文構造の要約や検索に応用することが考えられる。

そもそもword2vecを始めとしたツールにより学習される単語の意味表現ベクトルは、要約や検索といった種々の自然言語処理の課題において、その精度向上のための足掛かりとして、意味を持つ最小単位である単語・形態素の数値表現を与えることを目的としているので、こうした処理がまた別な自然言語処理タスクの一助となるならこの上なく喜ばしいことである。
